\section{\textsc{Frühlingshafter Eiersalat}}

\subsection*{Zutaten für 2 Portionen:}

\begin{tabular}{p{7.5cm} p{7.5cm}}
	& \\
	\textbf{Für den Salat:} & \textbf{Für die Soße:} \\
	Verschiedene Blattsalate & 5EL frische Kräuter \\
	1TL Essig & 1 Knoblauchzehe \\
	1EL Olivenöl & 1 Scharlotte \\
	4 hartgekochte Eier & 1TL Senf \\
	Salz und Pfeffer & 3EL Essig \\
	& 4EL Olivenöl \\
\end{tabular}

\subsection*{Serviervorschlag:}

\includegraphics[width=\textwidth]{img/ph.jpg} \cite{fruehlingeiersalat}

\subsection*{So geht's:}

\begin{tabular}{p{15cm}}
	\\
	Eier halbieren und das Eigelb vom Eiweiß trennen. Den Knoblauch und die Schalotten schälen und klein hacken.\\
	Das Eigelb, den Senf, den Knoblauch und die Schalotten in einen Messbecher geben.\\
	Den Essig und das Öl dazugeben und ein paar Minuten ziehen lassen.\\
	In der Zwischenzeit die Kräuter mit einem Wiegemesser fein hacken. Und in zu den anderen Zutaten hinzufügen.\\
	Das Eiweiß ebenfalls klein würfen und mit dazu geben.\\
	Alles fein pürieren und abschmecken.\\
	Die Salate putzen und klein schneiden.\\
	Öl, Essig und Gewürze hinzufügen und 10min ziehen lassen.\\
	In einer großen Schüssel anrichten und mit dem restlichen Ei garnieren.\\
	Baguette oder frische Brötchen dazu servieren.\\
	\textbf{Tipp:}
	Der Salat macht sich super zur Osterzeit. Es bleiben dort öfter mal gekochte Eier übrig, die sich so super verarbeiten lassen.
\end{tabular}

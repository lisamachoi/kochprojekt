\documentclass[10pt,a4paper,oneside]{article}
 \renewcommand{\familydefault}{\sfdefault}
\usepackage[utf8]{inputenc}
\usepackage[german]{babel}
\begin{document}
\thispagestyle{empty}
\pagenumbering{gobble}
\title{Projektarbeit}
\author{Lisa Machoi}
\date{\today}
\maketitle
\newpage
\tableofcontents
\setcounter{page}{0}
\pagenumbering{arabic}
%
\newpage
\setcounter{page}{1}
\section{Rohkostsalat mit Apfel}
Zutaten für 4 Personen:
\begin{itemize}
	\item 1x Weißkohlkopf
	\item 4x Karotten
	\item 2x Äpfel
	\item 6EL Olivenöl
	\item 6EL Weißweinessig
	\item 2EL Salz
	\item Pfeffer, Zucker, Kümmel nach Geschmack
\end{itemize}
Den Weißkohl teilen und danach in dünne Streifen schneiden.
Wir empfehlen hier die Schnittform Julienne.
Danach die Äpfel und die Karotten schälen.
Den Apfel in Achtel teilen.
Salz und Essig hinzufügen und solange fest kneten, bis der Kohl einen eigenen Saft bildet.
Nun mit Pfeffer, Zucker und den anderen Gewürzen abschmecken.
Danach das Gemisch einige Zeit ziehen lassen und fertig ist der Rohkostsalat.
Guten Appetit!
%
\newpage
\section{Frühlingshafter Eiersalat}
%
\newpage
\section{Gurkensalat mit Dill}
%
\newpage
\section{Gewienertes Omlett}
%
\newpage
\section{Omlett ToMo}
%
\newpage
\section{Eierkuchen}
%
\newpage
\section{Blattsalat mit Lauch und Ei}
\end{document}
